\documentclass[12pt]{article}
\usepackage[margin=1in]{geometry}
\usepackage{listings}
\usepackage{amsmath}
\usepackage{amsfonts}
\usepackage{enumitem}  

\title{Leetcode Notes and Practice}
\author{Sebastian Pucher}

\begin{document}
\maketitle                             


\tableofcontents

\newpage

\section{Hashing}
\subsection{Valid Anagram}
\begin{enumerate}
  \item[] \textbf{Question:} Given two strings s and t, return true if the two strings are anagrams of each other, otherwise return false
    \begin{enumerate}
      \item[-] Create two separate dictionaries
      \item[-] Loop through one of the input strings, add key letter or letter freq.
      \item[-] If dict are the same, return true
    \end{enumerate}
\end{enumerate}

\subsection{Twosum}
\begin{enumerate}
  \item[] \textbf{Question:} Given an array of integers nums and an integer target, return the indices i and j such that nums[i] + nums[j] == target and i != j
    \begin{enumerate}
      \item[-] Use a hashmap to store the index of each number in the array as the \textit{value}
      \item[-] On each iteration, check first to see if the difference between the target val and the current num is already stored in the hashmap
      \item[-] If it is, then return the value at that key (the index), as well as the current index i
      \item[-] If it's not, then add the current number and index to the hashmap 
      \item[-] \textbf{Key Idea: } Always check the existence between the target and the current number as a key in the hashmap first!
    \end{enumerate}
\end{enumerate}

\section{Two Pointers}
\subsection{Valid Palindrome}
\begin{enumerate}
  \item[] \textbf{Question:} Given two strings s and t, return true if the two strings are anagrams of each other, otherwise return false
    \begin{enumerate}
      \item[-] Create two separate dictionaries
      \item[-] Loop through one of the input strings, add key letter or letter freq.
      \item[-] If dict are the same, return true
    \end{enumerate}
\end{enumerate}

\end{document}

